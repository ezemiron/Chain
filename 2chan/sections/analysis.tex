\section{Analysis of SIM data}

  Methods for analysis of SIM data are not deeply different from the
  methods used to to analyse data from typical widefield microscopy. The
  mathematical principles remain unchanged, only the pixel size has been
  reduced which may require a more careful thought.  The main issues with
  analysis of SIM data pertain to image pre-processing and image size.

  The first refers to image reconstruction and its validation and is similar
  in concept to validation one ought to perform to deconvolved widefield
  images. The later refers to the increased image size which require
  careful thought on the implementation of the analysis algorithms,
  increasingly powerful computers, or free time to wait for the
  computations to end.


  \subsection{Pre-processing}
    The output from SIM reconstruction is an image with negative values
    and floating point format.  These are valid values for an image and
    the typical mathematical operations involved in image analysis but
    the tools used to perform them are often unable to handle them,
    effectively forcing them to a conversion into an integer format.

    \todo{test in ImageJ if values are casted or rounded for the
          16bit conversion.}
    \todo{should we discuss meaning of negative values?}
    \todo{check if there's any reconstruction software that performs
          this reconstruction automatically and does not give the
          original reconstructed data}

    Such conversion is typically performed by simply casting all values
    into unsigned 16-bit integers (range 0--65535) which sets all negative
    values to zero. An alternative method is to use the mode of the pixel
    values.  This has the assumption that the largest group of pixels in
    the image are background pixels and that reconstructed pixel values
    are not intensity and require correction.

    \todo{is it worth discussing rescaling data? I guess it's technically
          correct but does anyone do it?}
    \todo{we should make reference to SIMcheck but that needs to get
          published first.}

    Whatever the method chosen, it is important to keep the original data.
    The conversion to integers leads to loss of information and is not
    required for the analysis, it is only a limitation of some tools.

    \begin{figure}
      \missingfigure{histogram display of the different scale methods}
    \end{figure}


  \subsection{Count of foci / Identification of objects}
    Among the most typical image analysis is counting the number of foci
    in a cell.  The techniques used to segment individual objects in other
    images will work equally well in SIM data.  They might actually work
    better  due to the increased resolving power.  \todo{I guess it is
    possible that polyclonal antibodies will fail for large complexes.}

    We have routinely analysed histone PTM foci and proteins from the NPC
    in SIM data by watershed, local maxima, and automatic threshold, none
    of which are new in the image analysis field.  Our method (source code
    available as supplementary data and online --- link to our github clone),
    follows the following logic:

    \begin{enumerate}
      \item Creation of nuclear mask on DAPI channel;
      \begin {enumerate}
        \item grayscale dilation to fill low-values / IC space;
        \item automatic threshold by Otsu's method based on the whole
              histogram.  It is important to not perform one per Z slice
              otherwise the top and bottom slices do not catch the cell
              ``caps'';
        \item fill object holes;
        \item filter of small objects;
      \end{enumerate}

      \item foci segmentation on each individual nuclear masks;
      \begin {enumerate}
        \item automatic threshold by Otsu's method to identify the nuclear
              background.  The histogram used is from the pixels in the
              nuclear mask and not the whole image otherwise there would be
              3 pixel populations: the image background, nuclear background,
              and foci.  \todo{check Antti Niemistö work, there may be
              alternative threshold algorithms that could be used instead}.
        \item set nuclear background to the image type minimum value;
        \item watershed of the image complement;
        \item set watershed lines to the image type minimum value;
        \item count the leftovers connected components.
      \end{enumerate}
    \end{enumerate}


  \subsection{Analysis of foci / objects}


  \subsection{Colocalization}
    Colocalization is likely to be the technique that will require more
    changes on image analysis but the problem is on the concept of
    colocalization itself.  One of the common methods is measuring
    co-occurrence which is defined by the presence in the same pixel of
    two signals.  However, as the resolving power increases, signals should
    never colocalize.  A perfect example of this is labelling of the same
    protein with two different antibodies to find they do not
    colocalize\todo{has Lothar published this?}.

    To make use of the same analysis method as before, one can simply perform
    a blurring of the super-resolution images.  However, this defeats the
    whole point of using super-resolution microscopy in the first place.

    A more elegant approach for the case of protein foci is the study of
    distances between the objects.  \todo{we should reference and summarize
    the paper on graph theory applied to microscopy once it gets published}.
